\documentclass[]{book}
\usepackage{lmodern}
\usepackage{amssymb,amsmath}
\usepackage{ifxetex,ifluatex}
\usepackage{fixltx2e} % provides \textsubscript
\ifnum 0\ifxetex 1\fi\ifluatex 1\fi=0 % if pdftex
  \usepackage[T1]{fontenc}
  \usepackage[utf8]{inputenc}
\else % if luatex or xelatex
  \ifxetex
    \usepackage{mathspec}
  \else
    \usepackage{fontspec}
  \fi
  \defaultfontfeatures{Ligatures=TeX,Scale=MatchLowercase}
\fi
% use upquote if available, for straight quotes in verbatim environments
\IfFileExists{upquote.sty}{\usepackage{upquote}}{}
% use microtype if available
\IfFileExists{microtype.sty}{%
\usepackage{microtype}
\UseMicrotypeSet[protrusion]{basicmath} % disable protrusion for tt fonts
}{}
\usepackage{hyperref}
\hypersetup{unicode=true,
            pdftitle={SEM and R},
            pdfauthor={Bill},
            pdfborder={0 0 0},
            breaklinks=true}
\urlstyle{same}  % don't use monospace font for urls
\usepackage{natbib}
\bibliographystyle{apalike}
\usepackage{color}
\usepackage{fancyvrb}
\newcommand{\VerbBar}{|}
\newcommand{\VERB}{\Verb[commandchars=\\\{\}]}
\DefineVerbatimEnvironment{Highlighting}{Verbatim}{commandchars=\\\{\}}
% Add ',fontsize=\small' for more characters per line
\usepackage{framed}
\definecolor{shadecolor}{RGB}{248,248,248}
\newenvironment{Shaded}{\begin{snugshade}}{\end{snugshade}}
\newcommand{\AlertTok}[1]{\textcolor[rgb]{0.94,0.16,0.16}{#1}}
\newcommand{\AnnotationTok}[1]{\textcolor[rgb]{0.56,0.35,0.01}{\textbf{\textit{#1}}}}
\newcommand{\AttributeTok}[1]{\textcolor[rgb]{0.77,0.63,0.00}{#1}}
\newcommand{\BaseNTok}[1]{\textcolor[rgb]{0.00,0.00,0.81}{#1}}
\newcommand{\BuiltInTok}[1]{#1}
\newcommand{\CharTok}[1]{\textcolor[rgb]{0.31,0.60,0.02}{#1}}
\newcommand{\CommentTok}[1]{\textcolor[rgb]{0.56,0.35,0.01}{\textit{#1}}}
\newcommand{\CommentVarTok}[1]{\textcolor[rgb]{0.56,0.35,0.01}{\textbf{\textit{#1}}}}
\newcommand{\ConstantTok}[1]{\textcolor[rgb]{0.00,0.00,0.00}{#1}}
\newcommand{\ControlFlowTok}[1]{\textcolor[rgb]{0.13,0.29,0.53}{\textbf{#1}}}
\newcommand{\DataTypeTok}[1]{\textcolor[rgb]{0.13,0.29,0.53}{#1}}
\newcommand{\DecValTok}[1]{\textcolor[rgb]{0.00,0.00,0.81}{#1}}
\newcommand{\DocumentationTok}[1]{\textcolor[rgb]{0.56,0.35,0.01}{\textbf{\textit{#1}}}}
\newcommand{\ErrorTok}[1]{\textcolor[rgb]{0.64,0.00,0.00}{\textbf{#1}}}
\newcommand{\ExtensionTok}[1]{#1}
\newcommand{\FloatTok}[1]{\textcolor[rgb]{0.00,0.00,0.81}{#1}}
\newcommand{\FunctionTok}[1]{\textcolor[rgb]{0.00,0.00,0.00}{#1}}
\newcommand{\ImportTok}[1]{#1}
\newcommand{\InformationTok}[1]{\textcolor[rgb]{0.56,0.35,0.01}{\textbf{\textit{#1}}}}
\newcommand{\KeywordTok}[1]{\textcolor[rgb]{0.13,0.29,0.53}{\textbf{#1}}}
\newcommand{\NormalTok}[1]{#1}
\newcommand{\OperatorTok}[1]{\textcolor[rgb]{0.81,0.36,0.00}{\textbf{#1}}}
\newcommand{\OtherTok}[1]{\textcolor[rgb]{0.56,0.35,0.01}{#1}}
\newcommand{\PreprocessorTok}[1]{\textcolor[rgb]{0.56,0.35,0.01}{\textit{#1}}}
\newcommand{\RegionMarkerTok}[1]{#1}
\newcommand{\SpecialCharTok}[1]{\textcolor[rgb]{0.00,0.00,0.00}{#1}}
\newcommand{\SpecialStringTok}[1]{\textcolor[rgb]{0.31,0.60,0.02}{#1}}
\newcommand{\StringTok}[1]{\textcolor[rgb]{0.31,0.60,0.02}{#1}}
\newcommand{\VariableTok}[1]{\textcolor[rgb]{0.00,0.00,0.00}{#1}}
\newcommand{\VerbatimStringTok}[1]{\textcolor[rgb]{0.31,0.60,0.02}{#1}}
\newcommand{\WarningTok}[1]{\textcolor[rgb]{0.56,0.35,0.01}{\textbf{\textit{#1}}}}
\usepackage{longtable,booktabs}
\usepackage{graphicx,grffile}
\makeatletter
\def\maxwidth{\ifdim\Gin@nat@width>\linewidth\linewidth\else\Gin@nat@width\fi}
\def\maxheight{\ifdim\Gin@nat@height>\textheight\textheight\else\Gin@nat@height\fi}
\makeatother
% Scale images if necessary, so that they will not overflow the page
% margins by default, and it is still possible to overwrite the defaults
% using explicit options in \includegraphics[width, height, ...]{}
\setkeys{Gin}{width=\maxwidth,height=\maxheight,keepaspectratio}
\IfFileExists{parskip.sty}{%
\usepackage{parskip}
}{% else
\setlength{\parindent}{0pt}
\setlength{\parskip}{6pt plus 2pt minus 1pt}
}
\setlength{\emergencystretch}{3em}  % prevent overfull lines
\providecommand{\tightlist}{%
  \setlength{\itemsep}{0pt}\setlength{\parskip}{0pt}}
\setcounter{secnumdepth}{5}
% Redefines (sub)paragraphs to behave more like sections
\ifx\paragraph\undefined\else
\let\oldparagraph\paragraph
\renewcommand{\paragraph}[1]{\oldparagraph{#1}\mbox{}}
\fi
\ifx\subparagraph\undefined\else
\let\oldsubparagraph\subparagraph
\renewcommand{\subparagraph}[1]{\oldsubparagraph{#1}\mbox{}}
\fi

%%% Use protect on footnotes to avoid problems with footnotes in titles
\let\rmarkdownfootnote\footnote%
\def\footnote{\protect\rmarkdownfootnote}

%%% Change title format to be more compact
\usepackage{titling}

% Create subtitle command for use in maketitle
\providecommand{\subtitle}[1]{
  \posttitle{
    \begin{center}\large#1\end{center}
    }
}

\setlength{\droptitle}{-2em}

  \title{SEM and R}
    \pretitle{\vspace{\droptitle}\centering\huge}
  \posttitle{\par}
    \author{Bill}
    \preauthor{\centering\large\emph}
  \postauthor{\par}
      \predate{\centering\large\emph}
  \postdate{\par}
    \date{2021-04-11}

\usepackage{booktabs}
\usepackage{amsthm}
\makeatletter
\def\thm@space@setup{%
  \thm@preskip=8pt plus 2pt minus 4pt
  \thm@postskip=\thm@preskip
}
\makeatother

\begin{document}
\maketitle

{
\setcounter{tocdepth}{1}
\tableofcontents
}
\hypertarget{sem-and-r}{%
\chapter{SEM and R}\label{sem-and-r}}

This is the starting point.

\hypertarget{intro}{%
\chapter{Introduction}\label{intro}}

The following R codes are from UCLA website ``\url{https://stats.idre.ucla.edu/r/seminars/rsem/}'' and I do not own the copyright of the R code. I wrote this R Markdown file for my own study purpose.

\textbf{Given this consideration, please do NOT distribute this page in any way.}

\hypertarget{definitions-basic-concepts}{%
\section{Definitions (Basic Concepts)}\label{definitions-basic-concepts}}

\hypertarget{observed-variable}{%
\subsection{Observed variable}\label{observed-variable}}

Observed variable: A variable that exists in the data (a.k.a item or manifest variable)

\hypertarget{latent-variable}{%
\subsection{Latent variable}\label{latent-variable}}

Latent variable: A variable that is constructed and does not exist in the data.

\hypertarget{exogenous-variable}{%
\subsection{Exogenous variable}\label{exogenous-variable}}

Exogenous variable: An independent variable either observed (X) or latent (\(\xi\)) that explains an engogenous variable.

\hypertarget{read-the-data-into-the-r-studio-environment.}{%
\section{Read the data into the R Studio environment.}\label{read-the-data-into-the-r-studio-environment.}}

It also calcuates the covariance matrix among all the variables in the data.

\begin{Shaded}
\begin{Highlighting}[]
\NormalTok{dat <-}\StringTok{ }\KeywordTok{read.csv}\NormalTok{(}\StringTok{"https://stats.idre.ucla.edu/wp-content/uploads/2021/02/worland5.csv"}\NormalTok{)}
\KeywordTok{cov}\NormalTok{(dat)}
\end{Highlighting}
\end{Shaded}

\begin{verbatim}
##        motiv harm stabi ppsych ses verbal read arith spell
## motiv    100   77    59    -25  25     32   53    60    59
## harm      77  100    58    -25  26     25   42    44    45
## stabi     59   58   100    -16  18     27   36    38    38
## ppsych   -25  -25   -16    100 -42    -40  -39   -24   -31
## ses       25   26    18    -42 100     40   43    37    33
## verbal    32   25    27    -40  40    100   56    49    48
## read      53   42    36    -39  43     56  100    73    87
## arith     60   44    38    -24  37     49   73   100    72
## spell     59   45    38    -31  33     48   87    72   100
\end{verbatim}

In the following, we conduct a simple linear regression.
\[sample \ variance-covariance \ matrix \hat{\sum} = \mathbf{S} \]

\begin{Shaded}
\begin{Highlighting}[]
\NormalTok{m1a <-}\StringTok{ }\KeywordTok{lm}\NormalTok{(read }\OperatorTok{~}\StringTok{ }\NormalTok{motiv, }\DataTypeTok{data=}\NormalTok{dat)}
\NormalTok{(fit1a <-}\KeywordTok{summary}\NormalTok{(m1a))}
\end{Highlighting}
\end{Shaded}

\begin{verbatim}
## 
## Call:
## lm(formula = read ~ motiv, data = dat)
## 
## Residuals:
##      Min       1Q   Median       3Q      Max 
## -26.0995  -6.1109   0.2342   5.2237  24.0183 
## 
## Coefficients:
##               Estimate Std. Error t value Pr(>|t|)    
## (Intercept) -1.232e-07  3.796e-01    0.00        1    
## motiv        5.300e-01  3.800e-02   13.95   <2e-16 ***
## ---
## Signif. codes:  0 '***' 0.001 '**' 0.01 '*' 0.05 '.' 0.1 ' ' 1
## 
## Residual standard error: 8.488 on 498 degrees of freedom
## Multiple R-squared:  0.2809, Adjusted R-squared:  0.2795 
## F-statistic: 194.5 on 1 and 498 DF,  p-value: < 2.2e-16
\end{verbatim}

\begin{Shaded}
\begin{Highlighting}[]
\KeywordTok{library}\NormalTok{(lavaan)}
\CommentTok{#simple regression using lavaan }
\NormalTok{m1b <-}\StringTok{   '}
\StringTok{  # regressions}
\StringTok{    read ~ 1 + motiv}
\StringTok{  # variance (optional)}
\StringTok{    motiv ~~ motiv}
\StringTok{'}

\NormalTok{fit1b <-}\StringTok{ }\KeywordTok{sem}\NormalTok{(m1b, }\DataTypeTok{data=}\NormalTok{dat)}
\KeywordTok{summary}\NormalTok{(fit1b)}
\end{Highlighting}
\end{Shaded}

\begin{verbatim}
## lavaan 0.6-8 ended normally after 14 iterations
## 
##   Estimator                                         ML
##   Optimization method                           NLMINB
##   Number of model parameters                         5
##                                                       
##   Number of observations                           500
##                                                       
## Model Test User Model:
##                                                       
##   Test statistic                                 0.000
##   Degrees of freedom                                 0
## 
## Parameter Estimates:
## 
##   Standard errors                             Standard
##   Information                                 Expected
##   Information saturated (h1) model          Structured
## 
## Regressions:
##                    Estimate  Std.Err  z-value  P(>|z|)
##   read ~                                              
##     motiv             0.530    0.038   13.975    0.000
## 
## Intercepts:
##                    Estimate  Std.Err  z-value  P(>|z|)
##    .read             -0.000    0.379   -0.000    1.000
##     motiv             0.000    0.447    0.000    1.000
## 
## Variances:
##                    Estimate  Std.Err  z-value  P(>|z|)
##     motiv            99.800    6.312   15.811    0.000
##    .read             71.766    4.539   15.811    0.000
\end{verbatim}

\hypertarget{sem}{%
\chapter{SEM}\label{sem}}

SEM and R

\bibliography{book.bib,packages.bib}


\end{document}
